\PassOptionsToPackage{unicode=true}{hyperref} % options for packages loaded elsewhere
\PassOptionsToPackage{hyphens}{url}
%
\documentclass[ignorenonframetext,]{beamer}
\usepackage{pgfpages}
\setbeamertemplate{caption}[numbered]
\setbeamertemplate{caption label separator}{: }
\setbeamercolor{caption name}{fg=normal text.fg}
\beamertemplatenavigationsymbolsempty
% Prevent slide breaks in the middle of a paragraph:
\widowpenalties 1 10000
\raggedbottom
\setbeamertemplate{part page}{
\centering
\begin{beamercolorbox}[sep=16pt,center]{part title}
  \usebeamerfont{part title}\insertpart\par
\end{beamercolorbox}
}
\setbeamertemplate{section page}{
\centering
\begin{beamercolorbox}[sep=12pt,center]{part title}
  \usebeamerfont{section title}\insertsection\par
\end{beamercolorbox}
}
\setbeamertemplate{subsection page}{
\centering
\begin{beamercolorbox}[sep=8pt,center]{part title}
  \usebeamerfont{subsection title}\insertsubsection\par
\end{beamercolorbox}
}
\AtBeginPart{
  \frame{\partpage}
}
\AtBeginSection{
  \ifbibliography
  \else
    \frame{\sectionpage}
  \fi
}
\AtBeginSubsection{
  \frame{\subsectionpage}
}
\usepackage{lmodern}
\usepackage{amssymb,amsmath}
\usepackage{ifxetex,ifluatex}
\usepackage{fixltx2e} % provides \textsubscript
\ifnum 0\ifxetex 1\fi\ifluatex 1\fi=0 % if pdftex
  \usepackage[T1]{fontenc}
  \usepackage[utf8]{inputenc}
  \usepackage{textcomp} % provides euro and other symbols
\else % if luatex or xelatex
  \usepackage{unicode-math}
  \defaultfontfeatures{Ligatures=TeX,Scale=MatchLowercase}
\fi
% use upquote if available, for straight quotes in verbatim environments
\IfFileExists{upquote.sty}{\usepackage{upquote}}{}
% use microtype if available
\IfFileExists{microtype.sty}{%
\usepackage[]{microtype}
\UseMicrotypeSet[protrusion]{basicmath} % disable protrusion for tt fonts
}{}
\IfFileExists{parskip.sty}{%
\usepackage{parskip}
}{% else
\setlength{\parindent}{0pt}
\setlength{\parskip}{6pt plus 2pt minus 1pt}
}
\usepackage{hyperref}
\hypersetup{
            pdftitle={Toolbox for analysis and prediction of peptide variant effects},
            pdfauthor={Group 3: Laura Sans, Felix Pacheco, Jacob Kofoed, Begoña Bolos Sierra},
            pdfborder={0 0 0},
            breaklinks=true}
\urlstyle{same}  % don't use monospace font for urls
\newif\ifbibliography
\usepackage{longtable,booktabs}
\usepackage{caption}
% These lines are needed to make table captions work with longtable:
\makeatletter
\def\fnum@table{\tablename~\thetable}
\makeatother
\usepackage{graphicx,grffile}
\makeatletter
\def\maxwidth{\ifdim\Gin@nat@width>\linewidth\linewidth\else\Gin@nat@width\fi}
\def\maxheight{\ifdim\Gin@nat@height>\textheight\textheight\else\Gin@nat@height\fi}
\makeatother
% Scale images if necessary, so that they will not overflow the page
% margins by default, and it is still possible to overwrite the defaults
% using explicit options in \includegraphics[width, height, ...]{}
\setkeys{Gin}{width=\maxwidth,height=\maxheight,keepaspectratio}
\setlength{\emergencystretch}{3em}  % prevent overfull lines
\providecommand{\tightlist}{%
  \setlength{\itemsep}{0pt}\setlength{\parskip}{0pt}}
\setcounter{secnumdepth}{0}

% set default figure placement to htbp
\makeatletter
\def\fps@figure{htbp}
\makeatother


\title{Toolbox for analysis and prediction of peptide variant effects}
\providecommand{\subtitle}[1]{}
\subtitle{22100 - R for Bio Data Science}
\author{Group 3: Laura Sans, Felix Pacheco, Jacob Kofoed, Begoña Bolos Sierra}
\date{Spring 2020}

\begin{document}
\frame{\titlepage}

\begin{frame}

\begin{block}{Project aims}

\begin{itemize}
\tightlist
\item
  Produce R script machine learning toolbox for protein and peptide
  bio-activities.
\item
  Features:

  \begin{itemize}
  \tightlist
  \item
    Support for both sequence or variant input.
  \item
    Support for several sequence encoders.
  \item
    Support for sequence based calculations.
  \item
    Support for several models.
  \item
    Visualization options.
  \end{itemize}
\end{itemize}

\end{block}

\begin{block}{R script overview 1}

\includegraphics[width=5.20833in,height=\textheight]{project_organisation.png}

\begin{itemize}
\tightlist
\item
  00\_do\_it.R

  \begin{itemize}
  \tightlist
  \item
    Main program and loading of packages
  \end{itemize}
\item
  01\_load.R

  \begin{itemize}
  \tightlist
  \item
    Load: Encoding matrices, peptide and protein variant information and
    associated bio data
  \item
    Save: All loaded data in .tsv format
  \end{itemize}
\end{itemize}

\end{block}

\begin{block}{R script overview 2}

\includegraphics[width=5.20833in,height=\textheight]{project_organisation.png}

\begin{itemize}
\tightlist
\item
  02\_clean.R

  \begin{itemize}
  \tightlist
  \item
    Load: Load data from 01\_load.R
  \item
    Wrangle data: Remove NaN, fixes
  \item
    Save cleansed data in .tsv format
  \end{itemize}
\item
  03\_augment.R

  \begin{itemize}
  \tightlist
  \item
    Load data from 02\_clean.R
  \item
    Augment data: Calculate sequences, descriptors, properties
  \item
    Save augmente data in .tsv format
  \end{itemize}
\end{itemize}

\end{block}

\begin{block}{R script overview 3}

\includegraphics[width=5.20833in,height=\textheight]{project_organisation.png}

\begin{itemize}
\tightlist
\item
  04\_model\_i.R

  \begin{itemize}
  \tightlist
  \item
    Load augmented data
  \item
    Perform model fitting
  \item
    Predict unknowns
  \item
    Plotting and reporting
  \end{itemize}
\end{itemize}

\end{block}

\begin{block}{R script overview 4}

\includegraphics[width=5.20833in,height=\textheight]{project_organisation.png}

\begin{itemize}
\tightlist
\item
  99\_proj\_func.R

  \begin{itemize}
  \tightlist
  \item
    Sequence encoder
  \item
    Sequence generator
  \item
    \ldots{}
  \end{itemize}
\end{itemize}

\end{block}

\begin{block}{Data peptide and protein data sources}

\begin{itemize}
\tightlist
\item
  Ideas for sequence / variant effects:

  \begin{itemize}
  \tightlist
  \item
    \url{https://www.mavedb.org/} is a public repository for datasets
    from Multiplexed Assays of Variant Effect (MAVEs), such as those
    generated by deep mutational scanning (DMS) or massively parallel
    reporter assay (MPRA) experiments. RESTful API.
  \item
    Other scientific litterature\ldots{}
  \end{itemize}
\item
  Ideas for sequence encoding matrices

  \begin{itemize}
  \tightlist
  \item
    BLOSUM - Physicochemical and substitution matrix
  \item
    Z-scales - Physicochemical
  \item
    T-scales - Topological
  \item
    MSWHIM - 3D electrostatic potential
  \end{itemize}
\end{itemize}

\end{block}

\begin{block}{Machine learning toolbox}

\begin{itemize}
\item
  Ideas for supported machine learning framework:

  \begin{itemize}
  \tightlist
  \item
    Gaussian Process Regression.
  \item
    Artificial Neutral Network.
  \item
    ElasticNet Regression.
  \end{itemize}
\end{itemize}

\end{block}

\begin{block}{Distribution of tasks, week 18}

\begin{itemize}
\item
  Laura: Descriptors and function for translating, data visualization /
  exploration
\item
  Jacob: Sequence generation
\item
  Felix, Begoña: Machine learning tool boox
\item
  Project state EOW - functional preprocessing, first steps with machine
  learning toolbox
\end{itemize}

\end{block}

\begin{block}{Distribution of tasks, week 19}

\end{block}

\end{frame}

\begin{frame}{final presentation}
\protect\hypertarget{final-presentation}{}

next slides are for the final presentation

\begin{block}{Content}

\begin{itemize}
\tightlist
\item
  Introduction
\item
  Methods
\item
  Results
\item
  Discussion
\item
  Conclusion
\end{itemize}

\end{block}

\begin{block}{Introduction}

Prediction of protein-protein interactions (PPI) are a challenging task.

ML models allow to exploit the content of these PPI data sets.

The aim of this project is to create a toolbox to predict the biological
activity of these peptides with machine learning models.

\end{block}

\begin{block}{Methods - the data sets}

\begin{longtable}[]{@{}lllllll@{}}
\toprule
\begin{minipage}[b]{0.05\columnwidth}\raggedright
\strut
\end{minipage} & \begin{minipage}[b]{0.04\columnwidth}\raggedright
Protein\strut
\end{minipage} & \begin{minipage}[b]{0.14\columnwidth}\raggedright
Target\strut
\end{minipage} & \begin{minipage}[b]{0.12\columnwidth}\raggedright
Biological activity\strut
\end{minipage} & \begin{minipage}[b]{0.08\columnwidth}\raggedright
Species\strut
\end{minipage} & \begin{minipage}[b]{0.08\columnwidth}\raggedright
Num of variants\strut
\end{minipage} & \begin{minipage}[b]{0.30\columnwidth}\raggedright
Score\strut
\end{minipage}\tabularnewline
\midrule
\endhead
\begin{minipage}[t]{0.05\columnwidth}\raggedright
Data set 1\strut
\end{minipage} & \begin{minipage}[t]{0.04\columnwidth}\raggedright
BRCA1\strut
\end{minipage} & \begin{minipage}[t]{0.14\columnwidth}\raggedright
BARD1 RING domain\strut
\end{minipage} & \begin{minipage}[t]{0.12\columnwidth}\raggedright
Ubiquitin E3 activity\strut
\end{minipage} & \begin{minipage}[t]{0.08\columnwidth}\raggedright
\emph{H. sapiens}\strut
\end{minipage} & \begin{minipage}[t]{0.08\columnwidth}\raggedright
5610\strut
\end{minipage} & \begin{minipage}[t]{0.30\columnwidth}\raggedright
Y2H assays\strut
\end{minipage}\tabularnewline
\begin{minipage}[t]{0.05\columnwidth}\raggedright
Data set 2\strut
\end{minipage} & \begin{minipage}[t]{0.04\columnwidth}\raggedright
ERK2\strut
\end{minipage} & \begin{minipage}[t]{0.14\columnwidth}\raggedright
Small molecule (SCH772984)\strut
\end{minipage} & \begin{minipage}[t]{0.12\columnwidth}\raggedright
Resistance to drugs\strut
\end{minipage} & \begin{minipage}[t]{0.08\columnwidth}\raggedright
\emph{H. sapiens}\strut
\end{minipage} & \begin{minipage}[t]{0.08\columnwidth}\raggedright
6810\strut
\end{minipage} & \begin{minipage}[t]{0.30\columnwidth}\raggedright
Drug sensitivity assays. Calculation of cell availability\strut
\end{minipage}\tabularnewline
\begin{minipage}[t]{0.05\columnwidth}\raggedright
Data set 3\strut
\end{minipage} & \begin{minipage}[t]{0.04\columnwidth}\raggedright
LDLRAP1\strut
\end{minipage} & \begin{minipage}[t]{0.14\columnwidth}\raggedright
OBFC1\strut
\end{minipage} & \begin{minipage}[t]{0.12\columnwidth}\raggedright
Protein translation\strut
\end{minipage} & \begin{minipage}[t]{0.08\columnwidth}\raggedright
\emph{H. sapiens}\strut
\end{minipage} & \begin{minipage}[t]{0.08\columnwidth}\raggedright
6385\strut
\end{minipage} & \begin{minipage}[t]{0.30\columnwidth}\raggedright
Y2H assays\strut
\end{minipage}\tabularnewline
\begin{minipage}[t]{0.05\columnwidth}\raggedright
Data set 4\strut
\end{minipage} & \begin{minipage}[t]{0.04\columnwidth}\raggedright
Pab1\strut
\end{minipage} & \begin{minipage}[t]{0.14\columnwidth}\raggedright
el4FG1\strut
\end{minipage} & \begin{minipage}[t]{0.12\columnwidth}\raggedright
Translation initiation\strut
\end{minipage} & \begin{minipage}[t]{0.08\columnwidth}\raggedright
\emph{S. cereviseae}\strut
\end{minipage} & \begin{minipage}[t]{0.08\columnwidth}\raggedright
1340\strut
\end{minipage} & \begin{minipage}[t]{0.30\columnwidth}\raggedright
Y2H assays\strut
\end{minipage}\tabularnewline
\bottomrule
\end{longtable}

\end{block}

\begin{block}{Methods - the data sets}

\end{block}

\begin{block}{Methods}

Dataset used libraries used data cleaning and data wrangling data visu
(only mention it, more in results) modelling (how, train set, test set,
models regression, ann\ldots{})

\end{block}

\begin{block}{Results}

data visu (some plots to show the distribution of the initial data,
LAURA WILL DO IT) basic statistics for the data we have (LAURA WILL DO
IT) models results

\end{block}

\begin{block}{Discussion}

\end{block}

\begin{block}{Conclusion}

\end{block}

\begin{block}{References}

\includegraphics[width=3.125in,height=\textheight]{file.png}

\begin{itemize}
\tightlist
\item
  \textbf{Data set 1}: L. M. Starita, D. L. Young, et al.
  \emph{Massively parallel functional analysis of brca1 ring
  domainvariants}, Genetics, vol.~200, no. 2, pp.~413--422, 2015. 
\item
  \textbf{Data set 2}: L. Brenan, A. Andreev, et al. \emph{Phenotypic
  characterization of a comprehensive set of missense mutants}. Cell
  Reports, vol.~17, no. 4, pp.~1171--1183, 2016. 
\item
  \textbf{Data set 3}: A deep mutational scan of LDLRAP1 based on a Y2H
  assay with the interactor OBFC1.
  {[}\url{https://www.mavedb.org/scoreset/urn:mavedb:00000036-a-1/}{]} 
\item
  \textbf{Data set 4}: D. Melamed, D. L. Young, et al. \emph{Combining
  natural sequence variation withhigh throughput mutational data to
  reveal protein interaction sites}. PLOS Genetics, vol.~11, no.
  2,pp.~1--21, 2015. 
\end{itemize}

\end{block}

\end{frame}

\end{document}
